\documentclass[a4paper,USenglish,cleveref, autoref]{lipics-v2019}
% \documentclass[a4paper,oribibl,envcountsame]{llncs}
% \pagestyle{plain}
% \usepackage[T1]{fontenc}
% \usepackage[misc]{ifsym}
 \usepackage{amsmath}
 \usepackage{amssymb}
% \usepackage{booktabs}
% \usepackage{bussproofs}
% \usepackage{cite}
% \usepackage{color}
% \usepackage{enumitem}
%\let\vec=\undefined
%\usepackage{mdsymbol}  %%% indirectly needs texlive-generic-extra
 \usepackage{stmaryrd}
% \usepackage{subfig}
% \usepackage{textcomp}
% \usepackage{url}
% \usepackage{xspace}
% \usepackage{mathtools}
% \usepackage{todonotes}
% \usepackage{enumitem}

\usepackage{cprotect}
\usepackage{listings}
\usepackage{lstautogobble}

\usepackage{relsize}

% \usepackage{array}
% \newcolumntype{x}[1]{>{\centering\arraybackslash\hspace{0pt}}b{#1}}
% \usepackage{diagbox}
% \usepackage{siunitx}

\usepackage{tikz}
\usetikzlibrary{positioning,shadows,decorations, arrows, shapes, decorations.markings,decorations.pathmorphing}

% \usepackage[final,tracking=true,kerning=true,spacing=true,stretch=10,shrink=10]{microtype}

%% BEGIN Times font
% \usepackage{mathptmx}
% \usepackage[scaled=.82]{beramono}
% \usepackage[scaled=.86]{helvet}
% \DeclareSymbolFont{letters}{OML}{txmi}{m}{it}
%% END Times font

\newcommand\rulename[1]{\ensuremath{\mathsf{#1}}}

% \usepackage[
%    a4paper,
%    pdftex,
%    pdftitle={Isabelle LLVM},
%    pdfauthor={Peter Lammich},
%    pdfkeywords={},
%    pdfborder={0 0 0},
%    draft=false,
%    bookmarksnumbered,
%    bookmarks,
%    bookmarksdepth=2,
%    bookmarksopenlevel=2,
%    bookmarksopen]{hyperref}

% \hyphenation{Isa-belle Nieuw-en-huis meta-logic multi-set multi-sets}


\bibliographystyle{plainurl}% the mandatory bibstyle


% Not really meant for highlighting isabelle source, but for easily writing latex that looks like
% isabelle
% 
% keyword level 1 - isabelle outer syntax 
% keyword level 2 - isabelle inner syntax programming constructs (if, let, etc)
% keyword level 3 - standard constants (length, mod, etc)
% keyword level 4 - isabelle proof methods


% \newcommand{\lsem}{\ensuremath{\mathopen{[\![}}}
% \newcommand{\rsem}{\ensuremath{\mathclose{]\!]}}}

\lstdefinelanguage{isabelle}{
  morekeywords={theorem,theorems,corollary,lemma,lemmas,locale,begin,end,fixes,assumes,shows,and,class,
    constrains , definition, where, apply, done,unfolding, primrec, fun, using, by, for, uses, file,
    schematic_lemma, concrete_definition, prepare_code_thms, export_code, datatype, type_synonym, typedef, value,
    proof, next, qed, show, have, hence, thus, interpretation, fix, context, sepref_definition,is,export_llvm
 } ,
  morekeywords=[2]{rec, return, bind, foreach, if, then, else, do, let, in, res, spec, fail, assert, while, case, of, 
    check},
%  morekeywords=[3]{length,mod,insert},
%   morekeywords=[4]{simp,auto,intro,elim,rprems,refine_mono,refine_rcg},
  sensitive=True,
  morecomment=[s]{(\*}{\*)},
}


\DeclareTextCommand{\shortunderscore}{T1}{{$\lambda$}}1
  {\\\%}{{$\lambda$}}1
  {\\\$}{{$\mathbin{\,\$\,}$}}1
  {->}{{$\rightarrow$}}1
  {<-}{{$\leftarrow$}}1
  {<.}{{$\langle$}}1
  {.>}{{$\rangle$}}1
  {<=}{{$\le$}}1
  {>=}{{$\ge$}}1
  {<->}{{$\leftrightarrow$}}1
  {-->}{{$\longrightarrow$}}2
  {<-->}{{$\longleftrightarrow$}}1
  {=>}{{$\Rightarrow$}}1
  {==}{{$\equiv$}}2
  {==>}{{$\implies$}}2
  {<=>}{{$\Leftrightarrow$}}1
  {~=}{{$\ne$}}1
  {|}{{$\mid$}}1
  {-`}{{$\rightharpoonup$}}1
  {|`}{{$\restriction$}}1
  {!!}{{$\bigwedge$}}1
  {(}{{$($}}1
  {)}{{$)$}}1
  {\{}{{$\{$}}1
  {\}}{{$\}$}}1
  {[}{{$[$}}1
  {]}{{$]$}}1
  {[|}{{$\llbracket$}}1
  {|]}{{$\rrbracket$}}1
  {\\<lbrakk>}{{$\lsem$}}1
  {\\<rbrakk>}{{$\rsem$}}1
  {|-}{{$\vdash$}}1
  {|=}{{$\models$}}1
  {|->}{{$\mapsto$}}1
  {|_|}{{$\bigsqcup$}}1
  {...}{{$\dots$}}1
  {\\x}{{$\times$}}1
  {_0}{{${}_0$}}1
  {_1}{{${}_1$}}1
  {_2}{{${}_2$}}1
  {_3}{{${}_3$}}1
  {_4}{{${}_4$}}1
  {_5}{{${}_5$}}1
  {_6}{{${}_6$}}1
  {_7}{{${}_7$}}1
  {_8}{{${}_8$}}1
  {_9}{{${}_9$}}1
  {_L}{{${}_L$}}1
  {\\_n}{{${}_n$}}1
  {\\_i}{{${}_i$}}1
  {\\_j}{{${}_j$}}1
  {\\_x}{{${}_x$}}1
  {\\_y}{{${}_y$}}1
  {\\impl}{{${}_\dagger$}}1
  {^*}{{$^*$}}1
  {^k}{{$^k$}}1
  {^d}{{$^d$}}1
  {\\<^sup>*}{{$^*$}}1
  {\\<^sub>*}{{$_*$}}1
  {\\<^sub>A}{{$_A$}}1
  {\\<^sub>r}{{$_r$}}1
  {\\<^sub>a}{{$_a$}}1
  {:_i}{{$:_i$}}1
  {\\<A>}{{$\mathcal{A}$}}1
  {\\<O>}{{\sf o}}1
  {\\<Phi>}{{$\Phi$}}1
  {\\<Psi>}{{$\Psi$}}1
  {\\<sigma>}{{$\sigma$}}1
  {\\<cdot>}{{$\cdot$}}1
  {\\<in>}{{$\in$}}1
  {\\<le>}{{$\le$}}1
  {\\<noteq>}{{$\ne$}}1
  {\\<lambda>}{{$\lambda$}}1
  {\\<longrightarrow>}{{$\longrightarrow$}}1
  {\\<longleftrightarrow>}{{$\longleftrightarrow$}}1
  {\\<Rightarrow>}{{$\Rightarrow$}}1
  {\\<Longrightarrow>}{{$\Longrightarrow$}}1
  {\\<rightarrow>}{{$\rightarrow$}}1
  {\\<leftarrow>}{{$\leftarrow$}}1
  {\\<mapsto>}{{$\mapsto$}}1
  {\\<equiv>}{{$\equiv$}}1
  {\\<and>}{{$\wedge$}}1
  {\\<or>}{{$\vee$}}1
  {\\<And>}{{$\bigwedge$}}1
  {\\<Up>}{{$\Uparrow$}}1
  {\\<Down>}{{$\Downarrow$}}1
  {\\<Union>}{{$\bigcup$}}1
  {\\<up>}{{$\uparrow$}}1
  {\\<down>}{{$\downarrow$}}1
  {\\<times>}{{$\times$}}1
  {\\<forall>}{{$\forall$}}1
  {\\<exists>}{{$\exists$}}1
  {\\<nexists>}{{$\nexists$}}1
  {\\<union>}{{$\cup$}}1
  {\\<inter>}{{$\cap$}}1
  {\\<subset>}{{$\subset$}}1
  {\\<subseteq>}{{$\subseteq$}}1
  {\\<supset>}{{$\supset$}}1
  {\\<supseteq>}{{$\supseteq$}}1
  {\\<alpha>}{{$\alpha$}}1
  {\\<beta>}{{$\beta$}}1
  {\\<gamma>}{{$\gamma$}}1
  {\\alpha}{{$\alpha$}}1
  {\\beta}{{$\beta$}}1
  {\\gamma}{{$\gamma$}}1
  {\\<Gamma>}{{$\Gamma$}}1
  {\\<langle>}{{$\langle$}}1
  {\\<rangle>}{{$\rangle$}}1
  {\\<not>}{{$\neg$}}1
  {\\<box>}{{$\oblong$}}1
  {\\<bot>}{{$\bot$}}1
  {\\<top>}{{$\top$}}1
  {\\<notin>}{{$\notin$}}1
  {\\<guillemotright>}{{$\gg$}}1
  {\\in}{$\in$}1
  {\\and}{$\wedge$}1
  {\\or}{$\vee$}1
  {\\Phi}{{$\Phi$}}1
  {\\Psi}{{$\Psi$}}1
  {\\le}{{$\le$}}1
  {\\Up}{{$\Uparrow$}}1
  {\\Down}{{$\Down$}}1
  {>>}{{$\gg$}}1
  {>>=}{{${\gg}{=}$}}1
  {<*lex*>}{{$\times_{\sf lex}$}}1
  {\\<open>}{{\rm\guilsinglleft}}1
  {\\<close>}{{\rm\guilsinglright}}1
}

% \newcommand{\is}{\lstinline[language=isabelle,basicstyle=\normalsize\ttfamily\slshape]}
\newcommand{\is}{\lstinline[language=isabelle]}
\newcommand{\q}[1]{\mbox{\guilsinglleft{#1}\hspace{-.0pt}\guilsinglright}}
% \newcommand{\isai}[1]{\q{\lstinline[language=isabelle,basicstyle=\normalsize\ttfamily\slshape]{#1}}}
\cMakeRobust\q


\lstset{autogobble}


\ccsdesc[500]{Theory of computation~Program verification}
\ccsdesc[300]{Theory of computation~Logic and verification}
\ccsdesc[300]{Theory of computation~Separation logic}
% \ccsdesc[100]{General and reference}%TODO mandatory: Please choose ACM 2012 classifications from https://dl.acm.org/ccs/ccs_flat.cfm 

\keywords{Isabelle/HOL,LLVM,Separation Logic,Verification Condition Generator,Code Generation}%TODO mandatory; please add comma-separated list of keywords



\begin{document}


\title{Generating Verified LLVM from Isabelle/HOL}
% \titlerunning{Isabelle LLVM}

\author{Peter Lammich}{The University of Manchester}{peter.lammich@manchester.ac.uk}{}{}
\authorrunning {P. Lammich}
\Copyright{Peter Lammich}

% \author{Peter Lammich}
% \authorrunning {P. Lammich}
% \institute{
%   The University of Manchester, England
%   \email{peter.lammich@manchester.ac.uk}
% %   Technische Universit\"at M\"unchen, Munich, Germany \\
% %   \email{lammich@in.tum.de}
% }

\maketitle

\setcounter{footnote}{0}

\supplement{\url{http://www21.in.tum.de/~lammich/isabelle_llvm}}

\begin{abstract}
We present a framework to generate verified LLVM programs from Isabelle/HOL.
It is based on a code generator that generates LLVM text from 
a simplified fragment of LLVM, shallowly embedded into Isabelle/HOL.
On top, we have developed a separation logic, a verification condition 
generator, and an LLVM backend to the Isabelle Refinement Framework.

As case studies, we have produced verified LLVM implementations of binary search and the 
Knuth-Morris-Pratt string search algorithm. These are one order of magnitude faster than 
the Standard-ML implementations produced with the original Refinement Framework, and on par 
with unverified C implementations. Adoption of the original correctness proofs to the new LLVM backend was straightforward.

The trusted code base of our approach is the shallow embedding of the LLVM fragment 
and the code generator, which is a pretty printer combined with some straightforward compilation steps.

% We formalize the semantics of Isabelle-LLVM, an imperative language shallowly 
% embedded into Isabelle/HOL, designed to be easily translated to actual LLVM text.
% On top of Isabelle-LLVM, we have developed
% \begin{enumerate}
% \item a code generator, to translate from Isabelle-LLVM to actual LLVM text; 
% \item a verification condition generator, to reason about Isabelle-LLVM programs;
% \item a backend for the Isabelle Refinement Framework, to enable a high-level refinement based approach to developing verified LLVM programs.
% \end{enumerate}
% 
% As an example, we adopt an existing formalization of the Knuth-Morris-Pratt string search algorithm to our new framework.
% With only slight changes to the original formalization, we obtain a verified LLVM implementation, which runs faster and uses less memory 
% than the originally verified Standard ML implementation.
% 
% The trusted code base of our approach is the Isabelle-LLVM semantics, which we have engineered for conciseness and readability, 
% and the code generator that translates from Isabelle to LLVM text, which is merely a pretty printer. 
% The other tools that we developed cannot compromise the soundness of the approach, as they only generate theorems that are proved via Isabelle's kernel.
\end{abstract}


\section{Introduction}\label{sec:introduction}

% Current code generator technology in Isabelle and other interactive theorem provers is limited to emit high-level functional target 
% languages, which are inherently less efficient than low-level imperative languages such as C or LLVM~\cite{LLVM-manual}.
% This work presents a framework to generate verified LLVM programs from Isabelle/HOL~\cite{NPW02,NiKl14}.
% 
% We developed the Isabelle-LLVM language, an imperative language shallowly embedded into Isabelle/HOL.
% Isabelle-LLVM is designed to be easily translated into actual LLVM text, which is done by our code generator.
% On top of Isabelle/LLVM we build a separation logic and a verification condition generator, which enables us to verify
% basic data structures and algorithms. To simplify verification of more complex algorithms, we provide a stepwise refinement 
% approach, based on the Isabelle Refinement Framework~\cite{LaTu12,La13}.


The Isabelle Refinement Framework~\cite{LaTu12,La13,La15} features a stepwise refinement approach to verified algorithms, using the Isabelle/HOL theorem prover~\cite{NPW02,NiKl14}.
It has been successfully applied to verify many algorithms and software systems, 
among them LTL and timed automata model checkers~\cite{ELNN13,BrLa18,WiLa18}, network flow algorithms~\cite{LaSe19,LaSe16}, a SAT-solver certification tool~\cite{La17_CADE,La17_SAT}, and even a SAT solver~\cite{FBL18}.
Using Isabelle/HOL's code generator~\cite{HKKN13}, the verified algorithms can be extracted to functional languages like Haskell or Standard ML.
However, the code generator only provides partial correctness guarantees, i.e., termination of the generated code cannot be proved.
Moreover, the generated code is typically slower than the same algorithms implemented in C or Java. 

The original Refinement Framework~\cite{LaTu12,La13} could only generate purely functional code. 
The first remedy to the performance problem was to introduce array data structures that behave like functional 
lists on the surface, but are implemented by destructively updated arrays behind the scenes, similar to Haskell's now deprecated DiffArray.
While this gained some performance, the array implementation itself was not verified, such that we had to trust its correctness. 
Moreover, an array access still required a significant amount of overhead compared to a simple pointer dereference in C.

The next step towards more efficient verified implementations was the Sepref tool~\cite{La15}. 
It generates code for Imperative HOL~\cite{BKHEM08}, which provides a heap monad inside Isabelle/HOL, 
and a code generator extension to generate code that uses the stateful arrays provided by ML, or the heap monad of Haskell.
The Sepref tool performs automatic data refinement from abstract data types like maps or sets to concrete implementations like hash tables, which 
can be placed on the heap and destructively updated. 
Moreover, it provides tools~\cite{La16} to assist in the definition of new data structures, exploiting `free theorems'~\cite{Wad89} that it 
obtains from parametricity properties of the abstract data types.
%
Using Imperative HOL as backend, we gained some additional performance: For example, the GRAT tool~\cite{La17_CADE,La17_SAT} provides a 
verified checker for UNSAT certificates in the DRAT format~\cite{WHH14}. It is faster than the unverified state-of-the art 
checker \textsc{drat-trim}~\cite{WHH14}, which is written in C. However, the GRAT tool spends most of its run time in an unverified 
certificate preprocessor. Nevertheless, optimizing the verified part of the code is important: The very same technique was also implemented in Coq,
using purely functional data structures~\cite{CMS17,CHHKS17}. There, the verified code was actually the 
bottleneck\footnote{Later, the checker was rewritten in ACL2, also using imperative data structures~\cite{CHHKS17,HHKW17}}.

This paper presents a next step towards efficient verified algorithms: A refinement framework to generate verified code in LLVM intermediate representation~\cite{LLVM-manual} 
with total correctness guarantees. LLVM is an imperative intermediate language with a powerful and well-tested optimizing compiler.
We first formalize the semantics of Isabelle-LLVM, a simple imperative language shallowly embedded into Isabelle/HOL, 
and designed to be easily translated to actual LLVM text (\S\ref{sec:semantics}). 
On top of Isabelle-LLVM, we build a separation logic and a verification condition generator, which allows convenient reasoning about Isabelle-LLVM programs (\S\ref{sec:vcg}).
Finally, we modify the Sepref tool to target Isabelle-LLVM instead of Imperative/HOL (\S\ref{sec:auto_ref}), connecting 
the Refinement Framework to our LLVM code generator. This only affects the last refinement step, 
such that most parts of existing verifications can be reused. 
As case studies (\S\ref{sec:casestudies}), we verify a binary search algorithm and adopt an existing formalization~\cite{HeLa17} of
the Knuth-Morris-Pratt string search algorithm~\cite{KMP77}. The resulting LLVM code is significantly
faster than the corresponding Standard-ML code and on par with unverified C implementations.
The paper ends with the discussion of future work (\S\ref{sec:future_work}) and related work (\S\ref{sec:related_work}).
%
The Isabelle theories described in this paper are available as supplement material (URL displayed in paper header).



\section{Isabelle-LLVM}\label{sec:semantics}
\subsection{State Monad}
The basis of Isabelle-LLVM is a state-error monad, which we use to conveniently model the preconditions of instructions, their effect 
on memory, as well as arbitrary recursive programs. We define the algebraic data types:
\begin{lstlisting}
('a,'s) M = M (run: "'s => ('a,'s) mres")        ('a,'s) mres = NTERM | FAIL | SUCC  'a  's
\end{lstlisting}
An entity of type \q{\is{('a,'s) M}} contains a function \q{\is{run}} that maps a start state of type \q{\is{'s}} to 
a \emph{monad result} that indicates either nontermination, a failure, or a successful execution with a result of type \q{\is{'a}} and a new state.
We define the standard monad combinators:
\begin{lstlisting}
  "return x  = M (%s. SUCC x s)"                                get  = M (%s. SUCC s s)
  "fail       = M (%_. FAIL)"                                     set s = M (%_. SUCC () s)
  "bind m f  = M (%s. case run m s of SUCC x s => run (f x) s | r => r)"
  "assert \<Phi> = if \<Phi> then return () else fail"
\end{lstlisting}
That is, \q{\is{return x}} returns result \q{\is{x}} without changing the state, \q{\is{fail}} aborts the computation,
\q{\is{get}} returns the current state, and \q{\is{set s}} updates the current state.
Finally, \q{\is{bind m f}} first executes \q{\is{m}}, and then \q{\is{f}} with the result of \q{\is{m}}.
If \q{\is{m}} fails or does not terminate, the whole bind fails or does not terminate.  
The derived \q{\is{assert \<Phi>}} combinator can be conveniently used to abort the computation 
if some precondition is violated, e.g., on division by zero.

We use do-notation, i.e.\ \q{\is|do { x<-m; f x }|} is short for \q{\is{bind m (\<lambda>x. f x)}}.
Moreover, we define a flat chain complete partial order~\cite{Mark76} on \q{\is{mres}}, with \q{\is{\<bot> := NTERM}}.
For a monotonic function \q{\is{F :: ('a => ('b,'s) M) => 'a => ('b,'s) M}}, \q{\is{REC F}} is the least fixed point.
As functions defined using the monad combinators are monotonic by construction~\cite{Kr10}, 
we can define arbitrary recursive computations. The partial function package~\cite{Kr10} provides automation for monotonicity 
proofs and for defining simple recursive functions. Mutual recursion still requires some manual effort, though it could be automated, too.

% \subsubsection{Lenses}
% Lenses \cite{FGMPS07} are a convenient tool to express positions in structured data, which can be read from and written to.
% We define a notion of lenses that allows for the position to be invalid, depending on the data. 
% For example, the position ``nth element'' may be valid or not, depending on the element of the list. 
% A lens \q{\is{L}} is a pair of functions get and put, such that 
% \begin{lstlisting}
% get_put: "put L x s = Some s' ==> get L s' = Some x"
% put_get: "get L s = Some x ==> put L x s = Some s"
% put_put: "[|put L x s = Some s'; put L y s = Some s''|] ==> put L x s'' = Some s'"
% 
% put_indep: "put L y s != None <--> get L s != None"
% \end{lstlisting}
% 
% The type of a lens is \q{\is{'s ==> 'b}}, where \q{\is{'s}} (small) is the type at the specified position, and \q{\is{'b}} is the type of the whole.
% 
% The first three laws formalize the intuition of a position which we can get a value from, and put a value to:
% The value we get is the last one that we have put (\q{\is{get_put}}); if we put the value that is already there, nothing changes (\q{\is{put_get}}); 
% and if we first put one value, and then another one, the first value is overwritten by the second one (\q{\is{put_put}}).
% Finally, validity of a position does not depend on the value stored at this position (\q{\is{put_indep}}). 
% 
% Lenses can be composed. The lens \q{\is{L_1 \<bullet> L_2}} first focuses on a position using \q{\is{L_1}}, and then further focuses using \q{\is{L_2}}.
% \begin{example}
%   \q{\is{fstL :: 'a ==> 'a \x 'b}} is the lens that focuses on the first element of a pair, and \q{\is{idxL i :: 'a ==> 'a list}} is the lens that 
%   focuses on the $i$th position of a list.
%   That is, \q{\is{fstL\<bullet>idxL i :: 'a ==> 'a list \x 'b}} focuses on the ith position of the first element of a pair, e.g., we have
%   \q{\is{put (fstL\<bullet>idxL 2) 42 ([0,1,2,3], y) = Some ([0,1,42,3],y)}} and \q{\is{get (idxL 2) [] = None}}.
%   In the first example, we replace element 2 (indexing starts at 0) by 42, in the first element of the pair. 
%   In the second example, the position is invalid.
% \end{example}
% 
% We combine lenses and state monads by defining a \q{\is{zoom :: ('s ==> 'b) => ('a,'s,'f) M => ('a,'b,'f) M}} combinator.
% \q{\is{zoom L m}} executes \q{\is{m}} on the position focused by \q{\is{L}} of the state. 
% 
% Moreover, we use shortcut notations for applying lenses to the current state: 
% \q{\is{L:=x}} puts the value \q{\is{x}} into the part of the state focused by \q{\is{L}}, and \q{\is{use L}} gets the part of the state focused by \q{\is{L}}.
% 
% 


\subsection{Memory Model}
We use a high-level memory model that does not directly expose the bit-level representation of values and assumes an infinite supply of memory. 
The memory is modeled as a list of blocks. Each block is either deallocated, or it is a list of values.
A value is a pair of values, a pointer, or an integer. We model memory by the following data types\footnote{We have slightly simplified the presentation. 
The actual implementation defines the concepts memory, block, and value in a modular fashion, in order to ease future extensions.
% . The memory is parameterized over the block type, 
% the block type is parameterized over the value type, and the value type is parameterized over the primitive value type. This allows to 
% exchange parts of the memory model (e.g.\ include arbitrary structures instead of pairs) without breaking the rest of the memory model. 
% What we present here is the instance of our generic memory model that we use for this project.
}:
\begin{lstlisting}
memory = MEMORY (block list)                 block     = val list option
val = PAIR val val | PRIM primval              primval = PV_INT lint | PV_PTR rptr
\end{lstlisting}
%
Here, the type \q{\is{lint}} is a fixed bit width word type with a two's complement semantics, as used by LLVM, 
and pair corresponds to a 2-element structure in LLVM. 
The type \q{\is{rptr}} is either null or an address. An address is a path through the memory structure to a value:
\begin{lstlisting}
rptr  = NULL | ADDR nat nat (va_dir list)                      va_dir = PFST | PSND
\end{lstlisting}
An address consists of a \emph{block index}, a \emph{value index}, and a \emph{value address}, which is a list of directions 
to either descend into the first or the second value of a pair.

For the rest of this paper, we will use the state monad with a memory as state. Thus, we define the type \q{\is{'a llM = ('a,memory) M}}.
It is straightforward to define functions \q{\is{load :: rptr => val llM}} and \q{\is{put :: val => rptr => unit llM}} to read/write a value from/to a pointer, 
or fail if the pointer is invalid.
For the actual store function, we check that the structure of the value does not change, i.e. pairs remain pairs, pointers remain pointers, and words of width $w$ remain words of width $w$:
\begin{lstlisting}
  store x p = do { y <- load p; assert (vstruct x = vstruct y); put x p }
  $\text{where}$
  vstruct (PAIR a b) = VS_PAIR (vstruct a) (vstruct b)
  vstruct (PRIM (PV_PTR _)) = VS_PTR
  vstruct (PRIM (PV_INT w)) = VS_INT (width w) 
\end{lstlisting}

%
% 
% 
% Given a pointer \q{\is{p :: rptr}}, it is straightforward to define a lens \q{\is{ptr_L p :: val ==> memory}} that focuses on the pointed to value, 
% or fails if the pointer is invalid (null pointer, pointer to deallocated block, index out of bounds).
% Using this lens, we define a load and store function:
% \begin{lstlisting}
%   load p = use (ptr_L p)
%   store x p = do { y <- use (ptr_L p); assert (vstruct x = vstruct y); ptr_L p := x }
% \end{lstlisting}
% The load function simply returns the pointed-to value. The store function additionally checks that the structure of the value does not change, i.e.,
% pairs remain pairs, pointers remain pointers, and integers of width $w$ remain integers of width $w$. 
% This check is important in order to map our memory model to the actual LLVM memory model, which allocates an amount of memory matching the type of the value.
%
Similarly, we define an allocate and a free function:\\
\begin{minipage}[t]{.46\textwidth}
\begin{lstlisting}
  allocn v n = do {
    blocks <- get; 
    set (blocks@[Some (replicate n v)]);
    return (ADDR |blocks| 0 []) }
\end{lstlisting}
\end{minipage}
\hfill
\begin{minipage}[t]{.52\textwidth}
\begin{lstlisting}
  free (ADDR bi 0 []) = do {
    blocks <- get; 
    assert (bi < |blocks| \and blocks!bi ~= None);
    set (blocks[bi:=None]) }
  free _ = fail
\end{lstlisting}
\end{minipage}
Here, \q{\is{l_1@l_2}} concatenates two lists, \q{\is{|l|}} is the length of list \q{\is{l}}, \q{\is{l!i}} is the $i$th 
element of \q{\is{l}}, and \q{\is{l[i:=x]}} replaces the $i$th element of \q{\is{l}} by \q{\is{x}}.
The allocate function takes an initial value and a block size, appends a new block to the memory, and returns a pointer to the start of the new block (value index $0$, and value address $[]$).
The free function expects a pointer to the start of a block,
checks that this block is not already deallocated, and then deallocates the block by setting it to \q{\is{None}}.

\subsection{Towards a Shallow Embedding}
While we explicitly model values in memory by the type \q{\is{val}}, we model values in registers in a more shallow fashion:
We identify LLVM registers with Isabelle variables that have a type of shape \q{\is{T = T \x T | n word | T ptr}}. 
Here, \q{\is{\x}} is Isabelle's product type, \q{\is{n word}} is the $n$ bit word type from Isabelle's word 
library\cprotect\footnote{For convenient notation, we use the type \q{\is{n word}} as if it were a type depending on a variable $n$. 
Isabelle/HOL is not dependently typed. Instead, $n$ is actually a type variable with type-class \q{\is{len}}, which provides a 
function \q{\is{len_of :: 'a::len itself => nat}} to extract the length as a term.}, 
and \q{\is{'a ptr}} is a pointer with an attached phantom type for the value pointed to (\q{\is{'a ptr = PTR rptr}}).
%
For each type \q{\is{'a}} of shape \q{\is{T}}, we define the functions:
\begin{lstlisting}
to_val     :: "'a => val"            struct_of    :: "'a itself => vstruct"
from_val :: "val => 'a"             init          :: 'a
\end{lstlisting}
such that
\begin{lstlisting}
"from_val o to_val = id"                "vstruct (to_val x) = (struct_of TYPE('a))"
"to_val init = zero_initializer (struct_of TYPE('a))"
\end{lstlisting}
Here, \q{\is{TYPE('a) :: 'a itself}} reflects type \q{\is{'a}} into a term.
The functions \q{\is{to_val}} and \q{\is{from_val}} inject a T-shaped type \q{\is{'a}} into a value with structure \q{\is{struct_of TYPE('a)}}.
Moreover, \q{\is{init::'a}} corresponds to the all-zeroes value, i.e., the value where all pointers are null pointers, and all integers are $0$.

\subsection{Instructions}
In a next step, we define the instructions of Isabelle-LLVM. Each instruction is identified with an Isabelle constant. For example, the load instruction
is modeled by:
\begin{lstlisting}
  ll_load :: "'a ptr => 'a llM"
  ll_load (PTR p) = do {
    v <- load p;
    assert (vstruct v = struct_of TYPE('a));
    return (from_val v) }
\end{lstlisting}
It loads a value from the specified pointer, checks that its structure matches the expected type \q{\is{'a}}, and then 
converts the value to \q{\is{'a}}. 

For allocation and deallocation, we provide the instructions:
\begin{lstlisting}
  ll_malloc :: 'a itself => n word => 'a ptr llM            ll_free :: "'a ptr => unit llM"
\end{lstlisting}
Note that LLVM does not contain a heap manager. 
Instead, we assume that the generated code will be linked with the C standard library, 
and let the code generator produce calls to \q{\is{calloc}} and \q{\is{free}}.
%
We also define instructions to access the elements of a pair, to offset a pointer, and to advance a pointer into a pair.
The code generator maps these instructions to the corresponding LLVM instructions \q{\is{getelementptr}}, \q{\is{insertvalue}}, and \q{\is{extractvalue}}.

Integer instructions are defined on the \q{\is{n word}} type. For example, we define:
\begin{lstlisting}
  ll_udiv :: "n word => n word => n word llM"
  ll_udiv a b = do { assert (b ~= 0); return (a div b) }
\end{lstlisting}
where \q{\is{div}} is the unsigned division from Isabelle's word library.
Note the use of assertions to exclude undefined behavior, e.g., division by zero.

\subsection{Modeling Control Flow}\label{sec:modeling_ctrl_flow}
Next, we put together instructions to form procedure bodies. We only allow structured control flow 
via if-then-else, while, procedure calls, and sequential composition: The body of
a procedure is modeled by an Isabelle term of type \q{\is{'a llM}} and shape \q{\is{block}}, where
\begin{lstlisting}
block = do { var <- cmd; block } | return var
cmd = ll_<opcode> arg^* | proc_name arg^* | llc_if arg block block | llc_while block block
arg = var | number | null | init
\end{lstlisting}
% and \q{\is{llc_if :: 1 word => 'a llM => 'a llM => 'a llM}} is defined as
with
\begin{lstlisting}
llc_if :: 1 word => 'a llM => 'a llM => 'a llM
llc_if b t e = if b=1 then t else e

llc_while :: ('a => 1 word llM) => ('a => 'a llM) => 'a => 'a llM
llc_while b c s = do {ctd <- b s; llc_if ctd (do {s <- c s; llc_while b c s}) (return s)}
\end{lstlisting}
That is, a block is a list of commands whose results are bound to variables, terminated by a return instruction. 
A command is either an instruction, a procedure call, or an if-then-else or while statement.
The arguments of instructions and procedure calls, as well as the condition of an if-then-else statement, must be variables or 
constants (i.e., numbers, the null pointer, or a zero-initialized value). 
The condition of a while statement is modeled as a block returning a \q{\is{1 word}}, such that it can be re-evaluated prior to each loop iteration.
%
A program is represented by a set of (monomorphic) theorems of the shape
\q{\is{proc\_i x_1 ... x\_n = block}},
where the \q{\is{proc\_i}} are Isabelle functions, the \q{\is{x\_i}} are variables, and all free variables on the right hand side are among the \q{\is{x\_i}}.
% As our monad allows arbitrary recursive definitions, we can define arbitrary recursive programs.

\begin{figure}
\centering
\begin{minipage}[t]{.35\textwidth}
  \begin{lstlisting}
    fib:: 64 word => 64 word llM
    fib n = do {
      t <- ll_icmp_ule n 1;
      llc_if t (return n) (do {
        n_1 <- ll_sub n 1;
        a   <- fib n_1;
        n_2 <- ll_sub n 2;
        b   <- fib n_2;
        c   <- ll_add a b;
        return c
      }) }    
  \end{lstlisting}
  \caption{Isabelle-LLVM program}\label{fig:fib_isabelle}
\end{minipage}
% \hspace*{.04\textwidth}
\hfill
\begin{minipage}[t]{.57\textwidth}
  \begin{lstlisting}[language=LLVM, literate={}]
  define i64 @fib(i64 %x) {
    start:
      %t = icmp ule i64 %x, 1
      br i1 %t, label %then, label %else
    then:
      br label %ctd_if
    else:
      %n_1 = sub i64 %x, 1
      %a     = call i64 @fib (i64 %n_1)
      %n_2 = sub i64 %x, 2
      %b     = call i64 @fib (i64 %n_2)
      %c     = add i64 %a, %b
      br label %ctd_if
    ctd_if:
      %x1a = phi i64 [ %x, %then ], [ %c, %else ]
      ret i64 %x1a }  
  \end{lstlisting}
  \caption{Generated LLVM text}\label{fig:fib_llvm}
\end{minipage}
\end{figure}

\begin{example}\label{ex:fib_isabelle}
  Figure~\ref{fig:fib_isabelle} shows the Isabelle specification of a procedure named \q{\is{fib}}, 
  which takes a 64 bit word argument, and returns a 64 bit word. 
  Our semantics can be directly executed inside Isabelle. The following Isabelle command evaluates \q{\is{fib}} on the first few natural numbers, and an empty memory:
  \begin{lstlisting}
    value \<open>map (%n. run (fib n) (MEMORY [])) [0,1,2,3]\<close>
    (* output: [SUCC 0 (MEMORY []), SUCC 1 ..., SUCC 1 ..., SUCC 2 ...] *)
  \end{lstlisting}
\end{example}


\subsection{Code Generation}
The LLVM intermediate representation~\cite{LLVM-manual} is a strongly typed control flow graph (CFG) based intermediate language 
that uses single static assignment (SSA) form~\cite{CFRWZ91}.
A procedure is a list of basic blocks, the first block in the list being the entry point of the procedure. 
A basic block is a list of instructions, finished by a terminator instruction that determines the next basic block to execute (or to return from the current procedure).
Each non-void instruction defines a fresh register containing its result. A register can only be accessed in the part of the CFG that is dominated by its definition.
To transfer values from registers to other parts of the CFG, $\phi$-instructions are used. 
A $\phi$-instruction must be located at the start of a basic block. 
It lists, for each possible predecessor block, an accessible register in this predecessor block. 
The $\phi$-instruction evaluates to the value of the register from those predecessor block from which execution was actually transferred. 
The result of the $\phi$-instruction is bound to a fresh register, which can then be accessed from the current basic block.


It is straightforward to map an Isabelle-LLVM program to an actual LLVM program.
Each equation of the form \q{\is{proc x_1 .. x\_n = block}} is mapped to an LLVM function named \q{\is{proc}}. 
A block is mapped to a control flow graph. Instructions and procedure calls are directly mapped to LLVM instructions and calls.
An \q{\is{x <- llc_if b t e}} is translated to conditional branching, using a $\phi$-instruction to define the result register \q{\is{x}} when joining the control flow.
An \q{\is{x <- llc_while b c s}} is translated similarly. 


% 
% A command is mapped to a set of basic blocks and a register that holds the return value. 
% To concisely implement the translation, we use a builder pattern, that keeps track of the current basic block it is emitting instructions to.  
% The function \q{\is{build dst cmd}} translates a command, putting the result into the register named \q{\is{dst}}.
% The function \q{\is{build_block block}} translates a block, and returns the register that holds the result value of the block.
% For example, an \q{\is{llc_if}} command is translated as follows\footnote{The code displayed here is simplified, the actual code additionally 
% maintains the state of the builder and a mapping from Isabelle's variables to actual registers in LLVM. 
% We also omitted the case where the if-command returns unit, such that we do not create a $\phi$-node.}
% \begin{lstlisting}[language=ML]
%   fun build_if dst (llc_if b blk_t blk_e) = 
%   (* Omitted: Obtain fresh labels l_then, l_else, l_ctd_if *)
%   val _ = Builder.mk_cbr b l_then l_else
%   
%   val _ = Builder.open_bb l_then 
%   val r_then = build_block blk_t
%   val l_then' = Builder.mk_br l_ctd_if
%   
%   val _ = Builder.open_bb l_else 
%   val r_else = build_block blk_e
%   val l_else' = Builder.mk_br l_ctd_if
% 
%   val _ = Builder.open_bb b l_ctd_if
%   val _ = Builder.mk_phi dst [(r_then,l_then')] [(r_else,l_else')]
% \end{lstlisting}
% 
% The function takes a register \q{\is{dst}} where it should put the result to, and the Isabelle representation of the if command.
% We first obtain three fresh label names. Then, we close the current basic block with a conditional branch to the \q{\is{then}} and \q{\is{else}} label (\q{\is{mk_cbr}}).
% We then emit the code for the then and else part into the respective basic blocks, and close these basic blocks with unconditional branches to the \q{\is{ctd_if}} label.
% Finally, we open the \q{\is{ctd_if}} basic block, and emit a $\phi$-node to bind the result of the then and else part to the register \q{\is{dst}}.

\begin{example}
Figure~\ref{fig:fib_llvm} displays the output of our code generator for the \q{\is{fib}} constant displayed in Figure~\ref{fig:fib_isabelle}.
\end{example}


\subsubsection{Mapping the Memory Model}
Mapping the abstract memory model of Isabelle-LLVM to actual LLVM is slightly more involved.
% Some translations are slightly more involved.
For example, recall the \q{\is{ll_malloc :: 'a itself => n word => 'a ptr llM}} instruction. It has to be mapped to the function \q{\is{void* calloc(size_t, size_t)}} from
the C standard library. For this, we have to parameterize the code generator with the architecture dependent size of the \q{\is{size_t}} type. 
Next, we have to obtain the size of type \q{\is{'a}} and cast the \q{\is{n word}} parameter to \q{\is{size_t}}. Here, our code generator will refuse downcast, as this might result in bits being dropped. Finally, we have to cast the returned \q{\is{void*}} to the correct return type. 
Moreover, the \q{\is{calloc}} function returns \q{\is{null}} if not enough memory is available. In contrast, our semantics always returns a new block of memory.
We insert code to terminate the program in a defined way if it runs out of memory. The relation between our semantics and the actual LLVM program then 
becomes: Either the program terminates with an out-of-memory condition, or it behaves as modeled by the semantics. 
Our current implementation prints an error message and terminates the process with exit code $1$ if it runs out of memory.

A similar issue arises when comparing pointers: LLVM does not have instructions for pointer comparison. Instead, pointers have to be cast to integers, which can then be compared. However, this requires to know the bit-width of a pointer, which we cannot model in our semantics that admits unboundedly many different pointers.
Instead, we model the instructions \q{\is{ll_ptrcmp_eq}} and \q{\is{ll_ptrcmp_ne}}, and let the code generator generate the cast to integers and the integer comparison. 


\subsection{Preprocessing}
In the previous sections we have described the semantics of Isabelle-LLVM and its translation to actual LLVM.
However, Isabelle-LLVM programs have to adhere to a very restrictive shape (cf.~\S\ref{sec:modeling_ctrl_flow}), 
which makes them easy to map to actual LLVM code, but tedious to write directly.
Thus, we implement a preprocessor that tries to automatically transform user-specified equations to 
valid Isabelle-LLVM. While the preprocessing is highly incomplete, i.e., it cannot convert every equation to a well-shaped one, 
it works well in practice, allowing for concise specifications.
Note that the preprocessor \emph{proves} the new equations from the original ones.
Thus, errors in the preprocessor cannot affect soundness: Either, it fails to prove the equations, or it produces ill-shaped equations, 
which the code generator will reject.

The user specifies an initial set of constants, which must be instantiated to monomorphic types, i.e., must not contain any type variables.
For each constant, the preprocessor then gathers the defining equation, 
instantiates it to the actual monomorphic type of the constant, transforms it by inlining and fixed point unfolding,
and then repeats the process for any new constant occurring on the right-hand side of the transformed equation.
Note that a constant is identified by its name and type, such that a constant with the same name can occur multiple 
times in the final Isabelle-LLVM program. The code generator will disambiguate the names.
At the end, we have a set of monomorphic equations that define all constants that occur in the final program, and can be passed to the actual code generator.
We now describe the inlining and fixed point unfolding transformations.

% 
% 
% In this section, we describe the preprocessing techniques that we implemented to make programs simpler to write.
% The preprocessor starts with a set of user-specified constants. It then gathers the defining equations of these 
% constants, and, recursively, of all used constants, and transforms them to the shape expected by the code generator.. 
% Note that the preprocessor \emph{proves} the new equations from the original ones.
% Thus, errors in the preprocessor cannot affect soundness of the generated code: Either, it fails to prove the equations, or it produces ill-shaped equations, 
% which the code generator will reject.
% Our preprocessor includes three main functionalities: monomorphization, inlining, and fixed-point unfolding.

% \subsubsection{Monomorphization}
% We assume that the initial constants are instantiated to monomorphic types, i.e., do not contain any type variables.
% For each constant, the preprocessor then gathers the defining equation, 
% instantiates it to the actual monomorphic type of the constant, performs inlining and fixed point unfolding,
% and then repeats the process for the constants occurring on the right-hand side of the transformed equation.
% At the end, we have a set of monomorphic equations that define all constants that occur in the final program.
% We now describe the inlining and fixed point unfolding in more detail.

\subsubsection{Inlining}
Inlining first applies user defined rewrite rules and then flattens nested expressions, converting function calls to the shape 
\q{\is{r <- f x_1 ... x\_n}} or \q{\is{r <- return (f x_1 ... x\_n)}}, where the $x_i$ are either constants, variables, or \emph{monadic} arguments
of type \q{\is{... => _ llM}}. Subterms of type \q{\is{_ llM}} are recursively flattened.
We iterate the rewriting and flattening steps until a fixed point is reached. 

\begin{example}
Consider the following definition of the constant \q{\is{fib'}}:
\begin{lstlisting}
fib' :: "m word => m word llM"
fib' n = if n <= 1 then return n 
         else do { n_1 <- fib' (n - 1); n_2 <- fib' (n - 2); return (n_1 + n_2) }
\end{lstlisting}
When started with \q{\is{fib' :: 64 word => 64 word llM}}, the preprocessor automatically translates this equation 
to the equation displayed in Figure~\ref{fig:fib_isabelle}. During the translation, it uses the following inlining rules:
\begin{lstlisting}
  if b then c else t = llc_if (from_bool b) c t              return (a + b) = ll_add a b
  return (from_bool (a<=b)) = ll_icmp_ule a b               return (a - b) = ll_sub a b
\end{lstlisting}
Our default setup contains similar rules for the other operations, as well as rules to map tuples and case-distinctions over tuples to \q{\is{insertvalue}} and \q{\is{extractvalue}} instructions.
\end{example}

\subsubsection{Fixed-Point Unfolding}
The preprocessor generates recursive functions from fixed-point combinators. 
It examines the right hand side of an equation for patterns \q{\is{p}} for which it has an unfold rule of the form \q{\is{p = F p}}.
It then defines a new constant \q{\is{f x_1 ... x\_n = F (f x_1 ... x\_n)}}, where the \q{\is{x\_i}} are the free variables in the pattern \q{\is{p}}.
Finally, it replaces \q{\is{p}} by \q{\is{f x_1 ... x\_n}} in the equation. This way, specifications with fixed point combinators are automatically 
transformed to a set of recursive equations, as required by the code generator.

For example, the \q{\is{llc_while}} combinator is defined as a fixed point (cf.~\S\ref{sec:modeling_ctrl_flow}). 
Using its definition as an unfold rule, the preprocessor will automatically convert while loops into tail calls.
This allows for using while-loops without trusting their translation in the code generator. 
% However, to be efficient, we have to enable LLVM's tail call optimization, which will translate the function back into a loop.
A configuration option in our tool lets the user choose between direct while-loop translation or unfolding into a tail call.

% A standard example is the while loop combinator, which we define by the equation
% \begin{lstlisting}
%   llc_while b c s = do {ctd <- b s; llc_if ctd (do {s <- c s; llc_while b c s}) (return s)}
% \end{lstlisting}
% We use this equation as unfold rule, such that while loops are automatically converted to tail calls. Note that this 
% feature allows for the concise specifications of loops, without adding additional trusted code. 
% As LLVM has tail-call optimization, the compiled programs will be efficient.

\begin{example}\label{ex:euclid}
Consider the following program:
\begin{lstlisting}
euclid :: 64 word => 64 word => 64 word
euclid a b = do {
  (a,b) <- llc_while 
    (%(a,b) => ll_cmp (a ~= b))
    (%(a,b) => if (a<=b) then return (a,b-a) else return (a-b,b))
    (a,b);
  return a }
\end{lstlisting}
From this, the preprocessor proves the following two equations (before inlining):
\begin{lstlisting}
euclid a b = do {
    (a, b) <- euclid_0 (a, b);
    return a }
euclid_0 s = do {
    ctd <- case s of (a, b) => ll_cmp (a ~= b);
    llc_if ctd (do {
      s <- case s of (a, b) => if a <= b then return (a, b - a) else return (a - b, b);
      euclid_0 s
    }) (return s) }
\end{lstlisting}
That is, it defined a new constant \q{\is{euclid_0}} to replace the while loop by tail recursion.
\end{example}



\section{Verification Condition Generator}\label{sec:vcg}
The next step towards generating verified LLVM programs is to establish a reasoning infrastructure. 
In this section, we describe our separation logic~\cite{Rey02} based verification condition generator.
Note that, while applying complex operations on the proof state, at the end, our VCG conducts a proof 
that goes through Isabelle's inference kernel. Thus, bugs in the VCG cannot cause unsoundness.

\subsection{Separation Algebra}
The first step to obtain a separation logic is to define a separation algebra on a suitable abstraction of the memory.
A separation algebra~\cite{CHY07} is a structure with a zero, a disjointness predicate $a\#b$, 
and a disjoint union $a+b$.
Intuitively, elements describe parts of the memory. 
Zero describes the empty memory, $a\#b$ means that $a$ and $b$ describe disjoint parts of the memory,
and $a+b$ describes the memory described by the union of $a$ and $b$. 
For the exact definition of a separation algebra, we refer to~\cite{CHY07,KKB12}. We note that 
separation algebras naturally extend over functions, pairs, and option types. 

We abstract a value by a partial function from value addresses (\q{\is{va_dir list}}) to primitive values, 
such that the addresses in the domain of the function are independent, i.e., no address is the prefix of another address:
\begin{lstlisting}
  typedef aval = "{ m :: vaddr => 'a option. \<forall>va,va'\<in>dom m. va~=va' --> indep va va' }"
  val_\<alpha> :: val => aval
  val_\<alpha> (PRIM x) = [[] |-> x]
  val_\<alpha> (PAIR x y) = PFST \<cdot> val_\<alpha> x + PSND \<cdot> val_\<alpha> y
\end{lstlisting}
Here, \q{\is{[k |->v]}} is the partial function that maps \q{\is{k}} to \q{\is{v}}, and \q{\is{i \<cdot> a}} 
prepends the item \q{\is{i}} to all addresses in the domain of \q{\is{a}}. 
It is straightforward (though technically involved) to show that abstract values form a separation algebra, 
where the empty map is zero, maps are disjoint iff their domains are pairwise independent, and union merges two maps.

A natural abstraction of a block (\q{\is{val list}})
would be a function from indexes to abstract values, mapping invalid indexes to $0$. 
However, this abstraction does not contain enough information to reason about deallocation. 
In order to deallocate a block, we have to own the whole block. However, from the abstraction, we cannot infer the size of the block, and thus we cannot
specify an assertion that ensures that we own the whole block. A remedy (which the author has seen in~\cite{App14}) is to additionally 
abstract a block to its size. Thus, abstract blocks have the type \q{\is{ablock = (nat => aval) \x nat option}}. 
The option type is required to make the second elements of the tuples a separation algebra. We use the trivial separation algebra here, where
two elements are only disjoint if at least one of them is \q{\is{None}}. 
Finally, we define \q{\is{amemory = nat => ablock}}, and a function \q{\is{\<alpha> :: memory => amemory}} that abstracts memory
by a function from block indexes to abstract blocks, mapping deallocated or invalid indexes to zero.

\subsection{Basic Reasoning Infrastructure}
Predicates of type \q{\is{assn = amemory => bool}} are called \emph{assertions}.
The \emph{weakest precondition} of a program \q{\is{c :: 'a llM}}, a \emph{postcondition} \q{\is{Q :: 'a => assn}}, and a memory \q{\is{s}} is defined as:
\begin{lstlisting}
wp c Q s = (\<exists>r s'. run c s = SUCC r s' \and Q r (\<alpha> s'))
\end{lstlisting}
Intuitively, \q{\is{wp c Q s}} states that program \q{\is{c}}, if run on memory \q{\is{s}}, terminates successfully with the result \q{\is{r}}, and the abstraction of the new 
state \q{\is{s'}} satisfies \q{\is{Q}}.

For assertions \q{\is{P}} and \q{\is{Q}}, the \emph{separating conjunction} \q{\is{P*Q}} describes a memory that can be split into two disjoint parts described by \q{\is{P}} and \q{\is{Q}}, respectively:
\begin{lstlisting}
(P * Q) s = \<exists>s_1 s_2. s_1 # s_2 \and s = s_1 + s_2 \and P s_1 \and Q s_2
\end{lstlisting}
%
Validity of a \emph{Hoare triple} \q{\is|{P} c {Q}|} is defined as follows:
\begin{lstlisting}
|= {P} c {Q} = \<forall>F s. (P*F) (\<alpha> s) --> wp c (%r s'. (Q r * F) s') s
\end{lstlisting}
That is, if the memory can be split into a part described by the \emph{precondition} \q{\is{P}}, and a \emph{frame} described by \q{\is{F}},
then command \q{\is{c}} will succeed, and the new memory consists of a part described by the postcondition \q{\is{Q}} and the unchanged frame. 
Our Hoare triples satisfy the frame rule: \q{\is$|= {P} c {Q} ==> |= {P * F} c {\<lambda>r. Q r * F}$} for all \q{\is{F}}.

\subsection{Basic Rules}
Once we have set up the separation algebra and the abstraction function, we can prove Hoare triples for the basic operations of our memory model.
For example, we prove the following rules for \q{\is{allocn}} and \q{\is{free}}:
\begin{lstlisting}
|= {\<box>} allocn v n {%p. malloc_tag n p * range {0..<n} (%_. v) p}
|= {malloc_tag n p * \<exists>blk. range {0..<n} blk p} free p {%_. \<box>}
\end{lstlisting}
where \q{\is{\<box> = \<lambda>s. s=0}} describes the empty memory, 
\q{\is{malloc_tag n p}} asserts that \q{\is{p}} points to the beginning of a block, and the size field of this block's abstraction is \q{\is{n}}, 
and \q{\is{range I f p}} describes that for all \q{$i\in I$}, \q{$p+i$} points to value \q{$f~i$}.
Intuitively, \q{\is{allocn}} creates a block of size \q{$n$}, initialized with values \q{$v$}, and a tag. 
If one possesses both, the whole block and the tag, it can be deallocated by free. 
% Note that we use the size encoded in the tag to ensure that we possess the whole block. 
For the Isabelle-LLVM memory instructions, we obtain the following rules:
\begin{lstlisting}
|= {n~=0} ll_malloc TYPE('a) n {%p. range {0..<n} (%_. init) p * malloc_tag n p}
|= {range {0..<n} blk p * malloc_tag n p} ll_free p {%_. \<box>}
|= {pto x p} ll_load p {%r. r=x * pto x p}
|= {pto xx p} ll_store x p {%_. pto x p}
\end{lstlisting}
Here, \q{\is{pto x p}} describes that $p$ points to value $x$, and we write predicates as if they were assertions 
on the empty memory, e.g., \q{\is{n~=0}} instead of \q{\is{\<lambda>s. s=0 \<and> n~=0}}.
We prove similar rules for the other instructions.

\subsection{Automating the VCG}
In order to efficiently prove Hoare triples, some automation is required.
We provide a verification condition generator with a frame inference heuristics. 
The first step to prove a Hoare triple is to convert it to a proposition on weakest preconditions:
\begin{lstlisting}
  [|!!F s. STATE (P*F) s ==> wp c (%r s'. (Q r * F) s') s|] ==> |= {P} c {Q}
\end{lstlisting}
where \q{\is{STATE P s = P (\<alpha> s)}}.
In general, the VCG operates on subgoals of the form \q{\is{STATE P s ==> wp c Q s}}. 
It then iteratively performs one of the following steps\footnote{This is a simplified presentation. 
The actual VCG is an instantiation of a generic VCG framework that can be configured with various solvers, rules, and heuristics.}:
\begin{description}
  \item[simplification] Apply a rewrite rule to transform \q{\is{wp c Q s}} into some equivalent proposition. For example,
    binding is resolved by the rule:
    \begin{lstlisting}
      wp (do {x<-m; f x}) Q s = wp m (%x. wp (f x) Q) s
    \end{lstlisting}
  \item[rule] If there is a Hoare triple of the form \q{\is$|= {P'} c {Q'}$}, the VCG tries to infer a frame \q{\is{F}} 
  such that \q{\is{P |- P'*F}}, and replaces the goal by \q{\is{STATE (Q'*F) s' ==> Q s'}} for a fresh \q{\is{s'}}. 
  Here, \q{\is{P |- Q = \<forall>s. P s ==> Q s}} denotes entailment.
  \item[final] If the goal has the form \q{\is{STATE P s ==> Q s}} such that \q{\is{Q}} is not of the form \q{\is{wp _ _ _}}, a heuristics is used to prove 
  \q{\is{P |- Q}}.
\end{description}
The actual verification conditions are generated during frame inference and the final proof heuristics.
For example, the rule for \q{\is{ll_malloc}} requires to prove that the size operand is not zero. 
The VCG will try to prove these goals by a default tactic, and leave them to the user if this tactic fails.

\begin{example}\label{ex:euclid-proof}
Recall the function \q{\is{euclid :: 64 word => 64 word => 64 word llM}} from Example~\ref{ex:euclid}.
We prove the following Hoare triple:
\begin{lstlisting}
  |= {uint$_{64}$ a a\impl * uint$_{64}$ b b\impl * 0<a * 0<b} euclid a\impl b\impl {%r\impl. uint$_{64}$ (gcd a b) r\impl}
\end{lstlisting}
Here, \q{\is{uint$_{64}$ a a\impl}} states that \q{\is{a\impl::64 word}} is an unsigned integer with value \q{\is{a::int}}, 
where \q{\is{int}} is the type of (mathematical) integers in Isabelle, and \q{\is{gcd}} is Isabelle's greatest common divisor function. 
After annotating a suitable loop invariant, the VCG generates the following two verification conditions:
\begin{lstlisting}
  [| gcd x y = gcd a b; x ~= y; x <= y; ... |] ==> gcd x (y - x) = gcd a b
  [| gcd x y = gcd a b; \<not> x <= y; ... |] ==> gcd (x - y) y = gcd a b
\end{lstlisting}
These are straightforward to prove in Isabelle, e.g., using sledgehammer~\cite{BBP13}.
\end{example}

\subsection{Data Structures and Basic Refinement}
Recall Example~\ref{ex:euclid-proof}. The Hoare triple that is proved there first maps the 64 bit word arguments and results to mathematical integers,
and then phrases the correctness statement in terms of mathematical integers. 
This approach is often more feasible than stating correctness on the concrete implementation directly. 
In our case, we would have to define the concept of greatest common divisor for 64 bit words. In general, an algorithm often 
computes some function on abstract mathematical concepts like integers or sets, but has to implement these by concrete data structures like 64 bit words or hash-tables.
Thus, a concise way to specify the correctness statement is to first map the implementations back to the abstract concepts, and then state 
the actual correctness abstractly. 

In separation logic based reasoning, a data structure provides a \emph{refinement assertion} \q{\is{A x x\impl :: assn}}, which describes 
that the abstract value \q{\is{x}} is implemented by the concrete value \q{\is{x\impl}}. 
We define refinement assertions to implement integers and natural numbers by $n$ bit words, and to implement lists by blocks of memory.
On top of that, we define more complex data structures like dynamic arrays.
Note that new data structures can easily be added. In general, an implementation does not completely implement an abstract mathematical concept.
For example, $n$ bit words can only represent the integers \q{\is|sints n = $\{-2^{\text{n}-1}..<2^{\text{n}-1}\}$|}, and hash-tables can only represent finite sets. Thus, the rules for the operations generally come with additional preconditions. 
For example, the rule to implement subtraction on integers by subtraction on $n$ bit words is the following:
\begin{lstlisting}
|= {sint$_n$ a a\impl * sint$_n$ b b\impl * a-b \<in> sints n} ll_sub a\impl b\impl {%r\impl. sint$_n$ (a-b) r\impl}
  for a\impl b\impl :: n word and a b :: int
\end{lstlisting}
Here, \q{\is{sint$_n$}} implements mathematical integers by $n$-bit words.
Note that the postcondition does not mention the operands \q{\is{a,b}} again, though they are still valid after the operation. 
As \q{\is{sint$_n$}} is \emph{pure}, i.e., does not use the memory, our VCG will automatically add the corresponding assertions to the postcondition. 


\section{Automatic Refinement}\label{sec:auto_ref}
Our basic VCG infrastructure can be used to verify simple algorithms like \q{\is{euclid}} from Example~\ref{ex:euclid-proof}.
However, many complex algorithms have already been verified using the Isabelle Refinement Framework~\cite{LaTu12}.
It features a non-deterministic programming language with a refinement calculus and a VCG. 
It allows to express an algorithm using abstract mathematical concepts, and then refine it in multiple steps towards an efficient implementation.
The last step of a refinement is typically performed by the Sepref tool~\cite{La15}, which translates a program from the non-deterministic monad of the Refinement Framework 
into the deterministic heap monad of Imperative HOL~\cite{BKHEM08}, replacing abstract data types by concrete implementations. 
% It comes with some powerful automation and support for easily defining new data structures~\cite{La16}.
%
We have modified the Sepref tool to translate to Isabelle-LLVM's monad instead.
We only had to modify the translation phase. The preprocessing phases, which only work on the abstract program, remained unchanged.

The translation phase works by symbolically executing the abstract program, thereby synthesizing a structurally similar concrete program.
During the symbolic execution, the relation between the abstract and concrete variables is modeled by refinement assertions. 
The predicate \q{\is{hnr \<Gamma> m\impl \<Gamma>' R m}} means that concrete program \q{\is{m\impl}} implements abstract program \q{\is{m}},
where \q{\is{\<Gamma>}} contains the refinements for the variables before the execution, \q{\is{\<Gamma>'}} contains the 
refinements after the execution, and \q{\is{R}} is the refinement assertion for the result of \q{\is{m}}. For example, a \q{\is{bind}} is 
processed by the following rule:
\begin{lstlisting}[numbers=left]
[| hnr \<Gamma> m\impl \<Gamma>' R\_x m;
  !!x x\impl. hnr (R\_x x x\impl * \<Gamma>') (f\impl x\impl) ($R_x'$ x x\impl * \<Gamma>'') R\_y (f x);
  MK_FREE $R_x'$ free;
|] ==> hnr \<Gamma> (do {x\impl<-m\impl;r\impl<-f\impl x\impl; free x\impl; return r\impl}) \<Gamma>'' R\_y (do {x<-m; f x})
\end{lstlisting}
To refine \q{\is{x<-m; f x}}, we first execute \q{\is{m}}, synthesizing the concrete program \q{\is{m\impl}} (line~1).
The state after \q{\is{m}} is \q{\is{R\_x x x\impl * \<Gamma>'}}, where \q{\is{x}} is the result created by \q{\is{m}}.
From this state, we execute \q{\is{f x}} (line~2). The new state is \q{\is{$R_x'$ x x\impl * \<Gamma>'' * R\_y y y\impl}}, where \q{\is{y}} is the result of \q{\is{f x}}.
% , which is only implicit in the above rule.
Now, the variable \q{\is{x}} goes out of scope, such that it has to be deallocated. 
The predicate \q{\is&MK_FREE $R_x'$ free = \<forall>x x\impl. |= {$R_x'$ x x\impl} free x\impl {\<lambda>_. \<box>}&} (line~3) states that 
\q{\is{free}} is a deallocator for data structures implemented by refinement assertion \q{\is{$R_x'$}}. 
Note that the refinement for variable \q{\is{x}} may change: If \q{\is{f\impl x\impl}} overwrites \q{\is{x\impl}},
the refinement assertion for \q{\is{x}} will be changed to the special assertion \q{\is{invalid}}. 
The deallocator for \q{\is{invalid}} is simply a no-op.
%
Adding support for deallocators was the most substantial change we applied to the Sepref tool. Its original target 
language, Imperative HOL, is garbage collected, such that there is no need for explicit deallocation.

\subsection{Data Structure Library}
Once the basic Sepref tool is adapted, we can define data structures. 
Reusing the basic data structures from the original Sepref tool is not possible, as Imperative HOL 
uses arbitrary precision integers and algebraic data types, while we have only fixed width words and pairs.
Up to now, we have added the implementation of integers and natural numbers 
by $n$ bit words and some basic container data structures like dynamic arrays, bit-vectors, and min-heaps.
Thereby, we could reuse the existing infrastructure of the Sepref tool: For example,
there is support to automatically generate rules that also support refinement of the elements of a data structure, 
exploiting `free theorems'~\cite{Wad89} which stem from parametricity properties of the abstract types.

\section{Case Studies}\label{sec:casestudies}
To assess the usability of our approach, we have verified a binary search algorithm and 
the Knuth-Morris-Pratt string search~\cite{KMP77} algorithm. 
Both algorithms have also been verified with the original Sepref tool, such that we can compare the two approaches.

\subsection{Binary Search}
Binary search is a simple algorithm to find a value in a sorted array. 
Despite its simplicity, it has a history of flawed implementations\footnote{A buggy implementation in the Java Standard Library has gone undetected for nearly a decade~\cite{bs_flaw_blogpost}.}, making it a natural example for formal verification.

We start with a high-level specification: For a list \q{\is{xs}} and a value \q{\is{x}}, find the index of the first element greater or equal to \q{\is{x}}.
We define the following constant:
\begin{lstlisting}
  fi_spec xs x = spec i. i = find_index (%y. x<=y) xs
\end{lstlisting}
where \q{\is{find_index P xs}} is a standard list function that returns the index of the 
first element in \q{\is{xs}} that satisfies \q{\is{P}}, or \q{\is{length xs}} if there is no such element.

Next, we phrase the binary search algorithm in the Isabelle Refinement Framework:
\begin{lstlisting}
  bin_search xs x == do {
    (l,h) <- while
      (%(l,h). l<h) 
      (%(l,h). do {
        assert (l<length xs \and h<=length xs \and l<=h);
        let m = l + (h-l) div 2;
        if xs!m < x then return (m+1,h) else return (l,m)
      }) 
      (0,length xs);
    return l }"
\end{lstlisting}
It is a standard exercise to prove that the algorithm adheres to its specification:
\begin{lstlisting}
  bs_correct: sorted xs ==> bin_search xs x <= fi_spec xs x
\end{lstlisting}
Finally, we invoke our adapted Sepref tool: 
\begin{lstlisting}
  sepref_definition bs_impl [llvm_code] is bin_search
    :: "(larray${}_{64}$ sint${}_{64}$)$^k$ -> sint${}_{64}^k$ -> snat$_{64}$"
    unfolding bin_search_def [...] by sepref
  export_llvm bs_impl file "bin_search.ll"
  lemmas bs_impl_correct = bs_impl.refine[FCOMP bs_correct]
\end{lstlisting}
This produces an Isabelle-LLVM program \q{\is{bs_impl}}, exports it to actual LLVM text, and proves the refinement theorem \q{\is{bs_impl_correct}}:
\begin{lstlisting}
  (bs_impl, fi_spec) : [%(xs, _). sorted xs] (larray$_{64}$ sint$_{64}$)${}^k$ \x sint$_{64}^k$ -> snat$_{64}$
\end{lstlisting}
Here, \q{\is{snat${}_{w}$}} implements natural numbers by signed $w$-bit words\footnote{As LLVM's index operations are on signed words, 
  it's convenient to always implement sizes and indexes by signed types, even if they are natural numbers.}.
Moreover, \q{\is{larray${}_{w}$ A}} refines a list to an array and a $w$-bit length field, the elements of the list being refined by assertion \q{\is{A}}.
The notation \q{\is{[\<Phi>] $A_1^{k|d}$ \x ... \x $A_n^{k|d}$ -> R}} specifies a refinement with precondition \q{\is{\Phi}}, such that the 
arguments are refined by \q{\is{A_1 ... A\_n}} and the result is refined by \q{\is{R}}. 
The $\cdot^{k|d}$ annotations specify whether an argument is overwritten ($k$ for keep, $d$ for destroy).
While we use this notation a lot in the Refinement Framework, it is straightforward to prove a standard Hoare triple from it. By unfolding some definitions we get:
\begin{lstlisting}
  |= {larray$_{64}$ sint$_{64}$ xs xs\impl * sint$_{64}$ x x\impl * sorted xs }
     bs_impl xs\impl x\impl
     {%i\impl. \<exists>i. larray$_{64}$ sint$_{64}$ xs xs\impl * snat$_{64}$ i i\impl * i=find_index (%y. x<=y) xs}"
\end{lstlisting}
% TODO: Perhaps just annotate the bit-width!
That is, if we start with an array \q{\is{xs\impl}} representing the sorted list \q{\is{xs}}, and 
a 64-bit word \q{\is{x\impl}} representing the integer \q{\is{x}},
then the array still represents \q{\is{xs\impl}}, and the result \q{\is{i\impl}} represents a natural number \q{\is{i}}, which is equal to the correct index.

The Sepref tool implements mathematical integers by 64-bit words, proving absence of overflows.
This is only possible because the assertion in \q{\is{bin_search}} explicitly states that the indexes are in bounds. 
Moreover, note the expression \q{\is{l + (h-l) div 2}} that we used to compute the midpoint index. 
On mathematical integers, it is equal to \q{\is{(l+h) div 2}}. However, on fixed-width words, the latter may overflow, while the former does not\footnote{Exactly this 
overflow caused the infamous bug in the Java Standard Library~\cite{bs_flaw_blogpost}.}.

\subsection{Knuth-Morris-Pratt String Search}
Next, we regard the Knuth-Morris-Pratt (KMP) string search algorithm~\cite{KMP77}, a well-known 
linear time algorithm to find the index of the first occurrence of a string $s$ in a string $t$:
\begin{lstlisting}
"ss_spec s t = spec
  None => \<nexists>i. sublist_at s t i |
  Some i => sublist_at s t i \<and> (\<forall>ii<i. \<not>sublist_at s t ii))"
\end{lstlisting}
where \q{\is{sublist_at s t i}} specifies that list \q{\is{s}} occurs in list \q{\is{t}} at index \q{\is{i}}:
\begin{lstlisting}
  "sublist_at s t i = \<exists>ps ss. t = ps@s@ss \<and> i = length ps"
\end{lstlisting}


We have recently formalized KMP with the original Sepref tool~\cite{HeLa17}.
The adaption of the existing formalization was straightforward: 
In the abstract part, we had to explicitly add a few in-bounds assertions. Most of them were already contained implicitly in the original proof.
For the synthesis step, we only had to add setup for the fixed-width word types.
The result of the automatic synthesis is an Isabelle-LLVM program \q{\is{kmp_impl}}, and the theorem:
\begin{lstlisting}
(kmp_impl, ss_spec)
  : [%s t. |s| + |t| < $2^{63}$] (larray$_{64}$ sint$_{64}$)^k \x (larray$_{64}$ sint$_{64}$)^k -> snat_option$_{64}$
\end{lstlisting}
Here \q{\is{snat_option$_{64}$}} implements the type \q{\is{nat option}} by signed 64-bit words, mapping \q{\is{None}} to $-1$.


\subsection{Runtime}
We have compared our verified LLVM implementations to unverified C/C++ implementations of the same algorithms, 
as well as to the Standard ML (SML) implementations generated by the original Sepref tool. 
While we have implemented binary search in C ourselves, we used a publicly available code snippet~\cite{kmp_cpp} for 
KMP\footnote{One easily finds many C implementations of KMP, mainly differing in the loop structure. We tried to choose one that is close to our implementation.}.
The programs were compiled with MLton-$2018$~\cite{MLton} and clang-$6.0$~\cite{clang}, and run on a standard laptop machine (2.8GHz Quadcore i7 with 16MiB RAM).
%
Tables~\ref{tab:bs-runtimes} and \ref{tab:kmp-runtimes} display the 
results: The verified LLVM implementations are on par with the unverified C/C++ implementations, and an order of magnitude faster 
than the SML implementations. 

Isabelle's code generator uses arbitrary precision integers, which tend to be significantly slower than fixed-width integers.
The SML${}^*$ column shows the results when we manually replace the arbitrary precision integers by 64-bit integers in the generated code.
While this is unsound in general, it gives us a lower bound of what would be possible in SML with 
more elaborate code generator configurations\footnote{Fleury et al.~\cite{FBL18} have successfully experimented with such code generator tuning.}.
SML${}^*$ is significantly faster than the original SML, but still $1.5$ times slower than LLVM.

\begin{table}
\begin{minipage}{.49\textwidth}
  \centering  
  \begin{tabular}{c|cccc}
    $n/10^6$ & C    & LLVM & SML    & SML${}^*$ \\
    \hline
    $1$      & 121  & 100  & 1999   & 139      \\
    $2$      & 251  & 204  & 4209   & 289      \\
    $3$      & 379  & 304  & 6516   & 440      \\
    $4$      & 513  & 412  & 8843   & 600      \\
    $5$      & 635  & 514  & 11494  & 756      \\
    $6$      & 767  & 617  & 13646  & 917      \\
    $7$      & 908  & 726  & 16032  & 1076     \\
    $8$      & 1038 & 854  & 18421  & 1250     \\
    $9$      & 1162 & 945  & 20957  & 1409     \\
    $10$     & 1293 & 1045 & 23409  & 1564     \\
  \end{tabular}
  \caption{Time (ms) to search for the values $0,2,\ldots<5n$ in an array $[0,5,\ldots<5n]$.}\label{tab:bs-runtimes}
\end{minipage}
\hfill
\begin{minipage}{.49\textwidth}
  \centering  
  \begin{tabular}{c|cccc}
    $a$-$l$  & C++  & LLVM & SML  & SML${}^*$\\
    \hline    
    16-8     & 499 & 597  & 4616 & 918 \\
    16-64    & 511 & 598  & 4621 & 926 \\
    16-512   & 513 & 590  & 4573 & 909 \\
    32-8     & 453 & 551  & 4471 & 850 \\
    32-64    & 465 & 552  & 4523 & 857 \\
    32-512   & 463 & 544  & 4456 & 840 \\
    64-8     & 418 & 530  & 4433 & 803 \\
    64-64    & 420 & 531  & 4514 & 809 \\
    64-512   & 416 & 523  & 4411 & 799 \\
  \end{tabular}
  \caption{Time (ms) to run the $a$-$l$ benchmark suite from StringBench~\cite{stringbench}. 
    Here $a$ is the alphabet size, and $l$ the pattern size. The sample size is $3\cdot2^{20}$ characters. 
    The algorithm stops after finding the first match.}\label{tab:kmp-runtimes}
\end{minipage}
\end{table}

% 
% 
% 
% 
% 
% 
% 
% 
% 
% The same abstract algorithm can also be refined to imperative Standard ML code, using the original Sepref tool. 
% Moreover, we have implemented the algorithm directly in C. 
% While Isabelle's code generator uses arbitrary precision integers by default, there are extensions to support 
% machine integers~\cite{Native_Word-AFP}, but require some effort to be used with the Refinement Framework. 
% As a safe lower bound for the possible efficiency of these approaches, we have manually replaced the arbitrary precision integers by 64bit integers in the generated code\footnote{This is, of course, unsound in general.}.
% 
% Table~\ref{tab:bs-runtimes} shows the result of a benchmark ran on standard laptop hardware, using MLton~\cite{MLton} and Clang~\cite{Clang} as compilers.
% The unverified C implementation and the verified LLVM implementation are equally fast, while the SML code is an order of magnitude slower.
% When manually replacing arbitrary precision by 64bit integers (SML${}^*$), the resulting program is still slower by a factor of roughly $1.5$.
% 
% 
% 
% Table~\ref{tab:kmp-runtimes} shows the results of comparing the runtimes of
% the generated LLVM code against the original implementation (SML), the original implementation with manual replacements of arbitrary precision by fixed-width integers (SML${}^*$), and a C++ implementation taken from~\cite{kmp_cpp}\footnote{There are several implementations of KMP floating around the internet, implementing different variants of the algorithm. We tried to choose one that is similar to our variant.}.
% Again, LLVM is on par with C++, and one order of magnitude faster than SML, and roughly 1.5 times faster than SML${}^*$. 
% 

% 
% The code generator produces actual LLVM text from \q{\is{kmp_impl}}, which we compile and link with a simple command line interface written in C, 
% using \textsc{clang}~\cite{Clang}.
% Similarly, we generate Standard ML code from the original formalization, which we compile using \textsc{MLton}~\cite{MLton}.
% We compare the programs by running them on a large text that does not contain the search string: 
% The program compiled by LLVM is an order of magnitude faster than the one compiled by MLton, and uses a third of the memory.
% We leave more systematic testing to future work.
% 
% Table~\ref{tab:kmp-runtimes} shows the comparisons of the run-times. 

% 
% 
% We only had to slightly change the abstract part of the proof, in order to insert a few in-bounds assertions. 
% For example, when incrementing an index, we have to explicitly assert that it won't overflow. 
% As the invariants used in the original proof already contain this knowledge, adapting the proofs was straightforward.
% We then used the adapted Sepref tool to generate LLVM code for 64 bit indexes and 8 bit characters.
% 
% 
% We report on a case study on a slightly more complicated algorithm.
% We took an existing formalization~\cite{HeLa17} of the Knuth-Morris-Pratt string search~\cite{KMP77} in the Refinement Framework.
% It uses Sepref to generate Imperative HOL code, which is then translated to Standard ML by Isabelle's code generator.
% We only had to slightly change the abstract part of the proof, in order to insert a few in-bounds assertions. 
% For example, when incrementing an index, we have to explicitly assert that it won't overflow. 
% As the invariants used in the original proof already contain this knowledge, adapting the proofs was straightforward.
% We then used the modified Sepref tool to synthesize Isabelle-LLVM instead of Imperative HOL. 
% For this, we fixed the lengths and indexes to 64 bit integers, and the characters to 8 bit integers.
% For the search string and text we use a pair of an array and a length field.
% The result of the automatic synthesis is an Isabelle-LLVM program \q{\is{kmp_impl}}, and the theorem:
% \begin{lstlisting}
% (kmp_impl, kmp_SPEC) 
%   : [%s t. |s| + |t| < $2^{63}$] (larray sint)^k \x (larray sint)^k -> snat_option
% \end{lstlisting}
% Here, \q{\is{kmp_SPEC}} is the abstract specification of a string search algorithm, \q{\is{larray}} is 
% the refinement of a list by an array and a length, and \q{\is{snat_option}} implements the type \q{\is{nat option}} 
% by signed 64 bit integers, mapping \q{\is{None}} to $-1$. 
% The notation \q{\is{[\<Phi>] $A_1^{k|d}$ \x ... \x $A_n^{k|d}$ -> R}} specifies a refinement with precondition \q{\is{\Phi}}, such that the 
% arguments are refined by \q{\is{A_1 ... A\_n}} and the result is refined by \q{\is{R}}. 
% The $\cdot^{k|d}$ annotations specify whether the argument is overwritten ($k$ for keep, $d$ for destroy).
% While we use this notation a lot in the Refinement Framework, it is straightforward to prove a standard Hoare triple from it. By unfolding some definitions we get:
% \begin{lstlisting}
% |= {larray s s\impl * larray t t\impl * |s| + |t| < $2^{63}$}
%    kmp_impl s\impl t\impl
%      {%i. larray s s\impl * larray t t\impl 
%       * if i=-1 then \<nexists>i. sublist_at s t i 
%           else i>=0 \and sublist_at s t (unat i) \and (\<forall>j<unat i. \<not>sublist_at s t j)}
% \end{lstlisting}
% That is, our algorithm requires the length of the search string $s$ plus the length of the text $t$ to be less than $2^{63}$. 
% Then, it will return the index of the first occurrence of $s$ in $t$, or $-1$ if $s$ does not occur in $t$. 
% Moreover, the algorithm will neither change $s$ nor $t$, as expressed by the respective assertions in the postcondition.

% The code generator produces actual LLVM text from \q{\is{kmp_impl}}, which we compile and link with a simple command line interface written in C, 
% using \textsc{clang}~\cite{Clang}.
% Similarly, we generate Standard ML code from the original formalization, which we compile using \textsc{MLton}~\cite{MLton}.
% We compare the programs by running them on a large text that does not contain the search string: 
% The program compiled by LLVM is an order of magnitude faster than the one compiled by MLton, and uses a third of the memory.
% We leave more systematic testing to future work.
%TODO Add exact numbers.

\section{Future Work}\label{sec:future_work}
While our case studies only cover medium complex algorithms, we expect that our approach will scale to more complex algorithms, e.g.
model checkers~\cite{WiLa18,FBL18} and SAT solvers~\cite{FBL18}, which have already been formalized with the original refinement framework. 
While these formalizations use a combination of functional and imperative data structures, the LLVM backend only supports imperative data structures. 
We expect the necessary changes to be manageable, but non-trivial. 
In particular, the current Sepref tool only supports pure data structures to be nested in containers. In the Imperative HOL setting, we simply 
use functional data structures inside containers. For LLVM, nested container data structures currently require ad-hoc proofs on the separation logic level. We leave the lifting of Sepref to support nested imperative data structures to future work. 

Moreover, the refinement from arbitrary precision integers to fixed size integers was quite straightforward for our case studies, 
and we expect these refinements to be more complex in general.
We leave it to future work to explore this issue more systematically, and to provide semi-automated tools,
e.g.~along the lines of AutoCorres~\cite{GLAK14}.

Our code generator, as well as most standard code generators in theorem provers, 
translates from logic to target language code, implicitly identifying logical 
concepts with programming language concepts. This approach is simple, however, 
the translation algorithm and its implementation become part of the trusted code base.
More recently, code generators that translate into a deeply embedded semantics of the 
target language have been developed~\cite{MyOw14,HuNi18}. 
We leave a translation to a deep embedding of LLVM to future work, and note that a 
deep embedding will also enable more advanced control flow constructs like exceptions and breaking from loops, 
without significantly increasing the trusted code base.

Compared to actual LLVM, Isabelle-LLVM makes a few simplifying assumptions: 
We do not support floating point arithmetic, though this could be added, e.g.~based on Lei Yu's floating point formalization \cite{Yu13}.
Moreover, we only support two-element structures (pairs). This nicely fits Isabelle HOL's product datatype, and the nested structures 
resulting from longer tuples should not be a problem for LLVM's optimizer.
Also, we do not support concepts that are handy for program optimization, but not required for code generation, like poison values.
Isabelle-LLVM assumes an infinite supply of memory, and thus cannot assign a bit-size to pointers. 
This assumption helps us to retain a deterministic semantics, which is executable inside the theorem prover (cf.~Example~\ref{ex:fib_isabelle}). 
We plan to use this feature for systematic testing of our code generator against the actual LLVM compiler. 
A similar assumption is implicitly made for the stack, as our semantics permits arbitrarily deep recursive procedure calls.
We remedy this mismatch between semantics and reality by terminating the program in a defined way if it runs out of heap.
To protect against stack overflows, LLVM provides mechanisms like stack probing or split stack, which, however, require some effort to enable. 
We leave that to future work, and note that our generated code allocates no large blocks of memory on the stack. 
Thus, stack overflows are likely to hit the guard pages inserted by most operating systems, which will cause defined termination of the process.

Currently, we interface our generated LLVM code from C programs compiled by clang. However, the ABIs of C and LLVM only partially match, 
and some LLVM constructs cannot be expressed in C at all.
Currently, it is the user's responsibility to implement a correct header file. 
We plan to automatically generate header files and adapter functions to make the exported code accessible from C.


\section{Related Work}\label{sec:related_work}
This project would not have been possible without several independent Isabelle developments:
We use the Separation Algebra library~\cite{KKB12_afp,KKB12} as basis for our separation logic. We substantially extended this library by a 
frame inference heuristics, and formalized the extension of separation algebras over functions, products, and options. 
Moreover, we use Isabelle's machine word library~\cite{Word_Lib-AFP} to model the 2's complement arithmetic of LLVM. We slightly extended this library by adding a few lemmas. Finally, the Eisbach language~\cite{MMW16} was a great help for prototyping the verification condition generator, although most of the 
final VCG is now implemented directly in the more low-level Isabelle/ML.

The Vellvm project~\cite{ZNMZ12,ZNMZ13} verifies LLVM program transformations in Coq.
To be useful, e.g.~as backend for clang, they have to formalize a substantial fragment of LLVM. 
On the other hand, we can afford to formalize a simplified and abstract semantics that 
is just powerful enough to cover what Sepref generates.

We drew some of the ideas for our separation logic from the Verifiable C project~\cite{App14}, 
a Coq formalization of a separation logic on top of the CompCert C semantics~\cite{BL09}.

There exists various formalizations of low-level imperative languages, eg~\cite{Loch12,Watt18}. 
These are focused on specifying the semantics, and we are not aware of any complex 
algorithm verifications using these formalizations.
% CakeML~\cite{KTMK19} is a verified compiler from StandardML down to assembly code. 
% While it contains various low-level semantics, we are not aware direct correctness proofs

The DeepSpec project~\cite{deepspec} aims at a completely verified computation environment, 
down to machine code, including the operating system. 
This is much more ambitious than the work presented here, which stops at a (simplified) LLVM semantics.
For proving correct imperative programs, they have a separation logic based VCG for a fragment of C~\cite{App14,CBGD18},
which they apply to several small C programs, mainly for cryptographic algorithms. 


\section{Conclusions}
We have developed Isabelle-LLVM, a shallowly embedded imperative language designed to be easily translated to actual LLVM text. 
On top of this, we have built a verification 
infrastructure, and re-targeted the Sepref tool to connect the Refinement Framework to LLVM. 
As case studies, we have generated verified LLVM code for a binary search algorithm and the Knuth-Morris-Pratt string search algorithm.
Both implementations are an order of magnitude faster than the ones generated with the original Sepref tool, and on par with unverified C implementations.
The additional effort required to refine to LLVM instead of Standard ML was quite low.






% 
% Future work includes porting more existing verifications and data structures to Isabelle-LLVM, hopefully increasing their performance. 
% One challenge will be the limited support of nested container data structures by Sepref: The nested containers must either be 
% purely functional, or laborious ad-hoc proofs on the separation logic level are required. 
% As the former is not an option for (imperative) LLVM, we plan to add more support for nested imperative data structures to Sepref. 
% Other challenges will be the translation of sum-types (which are not supported by LLVM), and support for more 
% complex control flow statements.

% 
% 
% Future work will include the verification of more basic data structures, which will enable us to 
% port more existing verification projects to LLVM, hopefully increasing their performance. 
% One challenge will be the support of sum types: Unfortunately, union types have been discontinued in LLVM, such that we have to resort 
% to low-level bit-fiddling to support them. However, this will increase the complexity of the code generator. An alternative would be a more sophisticated memory model.
% The current shallow embedding of the control flow semantics also has the disadvantage that we cannot easily model more complex control flow like breaking from loops or exceptions, without significantly increasing the complexity of the code generator. To this end, we are already working on formalizing the semantics of 
% arbitrary control flow graphs, such that we can verify the translation of our control flow constructs to CFGs.
% Finally, Sepref does not support nested container data structures on the heap. 
% Currently, these have to be verified in an ad-hoc fashion using the underlying separation logic.
% For functional code, a common workaround is to use purely functional inner data structures. This is obviously not an option for LLVM. 
% We plan to investigate ways to add support for nested imperative data structures to Sepref.

{\footnotesize
\paragraph*{\footnotesize Acknowledgement}
We thank Maximilian P.\ L.\ Haslbeck and Simon Wimmer for proofreading and useful suggestions.
We received funding from DFG grant LA 3292/1 "Verifizierte Model Checker" and VeTSS grant "Formal Verification of Information Flow Security for Relational Databases".
}

\clearpage

% 
% 
% Discussion:
%   assumption on infinite supply of memory assumption, deterministic semantics
% 
% Future work:
%   handle stack overflow
%   systematically test semantics: execute in Isabelle and LLVM, compare results
%   start to develop data structures, port more algorithms to use LLVM as backend.
% 
% Mention used 3rd party libraries:
%   Word, Sep-Algebra, Eisbach
% 
% Related work
%   Vellvm, Compcert and its verification infrastructure
% 
  
  
  
% {\footnotesize
% 
% \paragraph{\footnotesize Acknowledgement}
% Max Haslbeck, Anders Schlichtkrull, and Mark Summerfield suggested textual
% improvements.
% %
% The work has received funding from the European Research Council
% under the European Union's Horizon 2020 research and innovation program
% (grant agreement No.\ 713999, Matryoshka).
% 
% }

% \bibliographystyle{splncs03}
\bibliography{bib}

\end{document}
